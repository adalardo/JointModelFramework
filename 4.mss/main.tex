% DOCUMENT CLASS
    % Change "letterpaper" to "a4" if you use a4 paper size
    \documentclass[a4,12pt]{article}

    \usepackage{titlesec} % Allows customization of titles
    \usepackage{authblk} % For multiple authors
    \usepackage{amsmath}
    \usepackage[utf8]{inputenc}
	\usepackage{setspace} % See \doublespacing command at the top of content.tex
    \usepackage{lineno} 	% See \linenumbers at the top of content.tex
    \usepackage{booktabs}
    \usepackage[skip=2.5\baselineskip]{caption} % to give more space between figs and captions
    \usepackage{listings} % allows me to insert code 
    \usepackage{pdfpages}

% REFERENCES
    \usepackage[firstinits=true, backend=biber, style=authoryear]{biblatex}
    \DeclareNameAlias{sortname}{last-first}

    \addbibresource{../../../BibTex_files/03_Chapter3.bib}

% GRAPHICS
    \usepackage{graphicx} % More advanced figure inclusion
    \usepackage{float} % For specifying table/figure locations, i.e. [ht!]
    \usepackage{changepage}
    \usepackage[table,xcdraw]{xcolor}

\title{Estimating interaction matrices for diverse, single-trophic communities}


\author[1]{Malyon D. Bimler}
\author[1]{Margaret M. Mayfield}
\author[2]{Trace E. Martyn}
\author[3]{Daniel B. Stouffer}


\affil[1]{School of Biological Sciences, The University of Queensland, St Lucia, Queensland, Australia.}
\affil[2]{School of Natural Resources and the Environment, The University of Arizona, Tucson, USA}
\affil[3]{Centre for Integrative Ecology, School of Biological Sciences, University of Canterbury, Christchurch, New Zealand}

%\setcounter{page}{54}

\begin{document}
\maketitle  
\newpage
% \linenumbers


\section{Abstract}
    
    \begin{enumerate}
    \item Understanding ecological communities requires an understanding of how the different species within it relate to one another. Those relationships are difficult to quantify in single trophic systems such as plant communities because they cannot be directly observed as in food webs or other species interaction networks, and must instead be inferred by other means. Current methods are inadequate for natural complex communities because they tend to rely on co-occurrence patterns (inaccurate) or intensive experimental designs (time-consuming, especially for many species). 
    \item We describe a general framework which allows us to estimate competitive and facilitative pair-wise interactions in diverse plant communities from empirical observations of performance (e.g. fecundity). This framework requires a multi-tiered approach to resolve all interactions. We first apply individual fitness models models to each species in order to estimate interactions between species commonly observed to co-occur. We then use the resulting interactions to approximate interactions which were not observed using a different set of assumptions. 
    \item This approach is modelled using a Bayesian statistical framework which allows us to estimate interactions coefficients despite high model complexity, as well as quantify the uncertainty around the resulting community matrix. We provide the model code in the STAN programming language. We illustrate the potential uses of this method by applying it to a case study on australian wildflowers and show how to scale the resulting interaction matrix into per-capita effects by integrating the above model framework in an annual plant population dynamic model. This particular application can be of wide use to both ecosystem management and improving our understanding of community diversity and stability.
    \end{enumerate}

% neighbour density-dependent performance models


\section{Introduction}
    
    \paragraph{}
    Interactions between species can affect growth rates and population abundances, for example as a predator eats a prey or two plants compete for resources. Together, interactions between many species influence the dynamics of the whole community in ways which can determine patterns of species richness and abundance. In order to understand how community dynamics behave when many species are involved, ecologists can represent communities as networks \parencite{Pimm1978} where species are nodes and linked by interactions. This approach provides a broad view of how multiple species relate to one another within a community \parencite{Kefi2015}. As a result, ecological networks have been observed to typically differ from randomly-assembled networks in important and varied ways, thus informing us on the biological processes structuring these communities \parencite{Dunne2002}. There is now a rich body of work describing the characteristics of food webs, plant-pollinator networks, and host-parasite interactions, but non-trophic interactions such as those occurring between plants are largely neglected \parencite{Ellison2019}. This is unfortunate, as recent work indicates that non-trophic interactions and plant diversity alter food web structure, stability and diversity functioning relationships \parencite{Hammill2015, Giling2019, Zhao2019, Miele2019}. Integrating plant and trophic networks remains a key challenge in improving our understanding of community dynamics and persistence \parencite{Godoy2018c}. 

    \paragraph{}
    Measuring non-trophic relationships in plant communities is difficult, in part because they cannot be directly observed as in food webs or other interaction networks, and must instead be inferred by other means. Arguably the most common approach to empirically quantifying interactions between a focal species and others involves measuring the effects of neighbours on a proxy for lifetime reproductive success (e.g. fecundity, growth rate) \parencite{Connell1961, Grace1990}. This can be done for example by growing focal species with neighbours of different identities and at varying densities, in the field or laboratory. The resulting phenomenological interaction strengths are not dependant on any specific mechanism and can be directly linked to changes in species abundances through the use of population dynamic models (for example with the Beverton-Holt model \textcite{Beverton1957, Levine2009}). This approach has been widely applied and has greatly improved our understanding of competition and its effects on biodiversity \parencite{Tilman1982, Chesson2000b, Levine2008, Adler2010, Mayfield2010a, Kraft2015} but remains limited to investigating interactions between few species, rather than a diverse community. The reason is two-fold: applying this method to a large number of species requires both large amounts of data and replication, as well as complex models which can be difficult to implement and run the danger of over-parameterisation. 

    \paragraph{} 
    A common alternative which is less demanding, and scalable to high-diversity systems is the use of association, or co-occurrence networks. Plant 'interactions' are inferred by measuring positive and negative spatial associations between species (e.g. \textcite{Saiz2011}), resulting in networks which capture a variety of factors including biotic interactions \parencite{Losapio2019}. This approach requires data which is much more easily obtainable and can naturally handle a large number of species. A wide range of packages are available in R to deduce association networks, alongside further tools and methods to better infer plant interactions from spatial distribution patterns \parencite{Keil2019}, for example by including environmental covariates. Spatial association networks however, have been found to remain poor predictors of species interactions and rarely match empirical estimates \parencite{Sander2017,Barner2018, Thurman2019, Blanchet2020}.


    \paragraph{}
    In this paper, we describe a framework to estimate non-trophic interactions within a diverse plant system, which can be extended to other horizontal communities. Because not all interactions which can occur may be observed, we use a multi-tiered model which allows us to estimate interactions between common as well as rarer species and apply it with Bayesian parameter estimation methods which are flexible to greater model complexity.  Our framework is applied to a case study of an annual wildflower plant community in Western Australia consisting of 22 focal species and up to 52 neighbouring species. We demonstrate how our method can integrate into and extend traditional plant population dynamic approaches to quantifying interactions by explicitly dealing with many species and allowing for both competitive and facilitative interactions. We illustrate how the resulting interaction matrices differs from a co-occurrence network on the same data, and suggest potential applications that make use of the rich information provided by species interaction networks.

    
\section{Methods}
    
    \paragraph{}
    This framework can be applied to any dataset of interacting species which meets the following criteria: 
    \begin{itemize}
        \item observations of focal individuals record a proxy for lifetime reproductive success (e.g. seed production, stem diameter growth, above-ground biomass)
        \item these observations also record the identity and abundance of neighbours within the interaction neighbourhood of each focal individual
        \item observations are replicated across several individuals of each focal species
    \end{itemize}
    In addition to these requirements, any experimental design or data which may reduce confounding effects between environment and competition will provide more accurate estimates of interaction strengths (e.g. thinning certain plots, recording environmental data known to affect reproductive success). 
    
    \subsection{Individual Fitness Model}
        
        \paragraph{}
        We begin by implementing an individual fitness model to each focal species $i$ which regresses the identity and abundance of neighbours $j$ ($j = 1, 2, 3, ...$) against the measured proxy for lifetime reproductive success $F_{i}$:
        
        \begin{equation}
        F_{i} = \beta_{i0} - \sum_{i}^{j} \beta_{ij} N_{j}
        \label{ifm}
        \end{equation}
        
        The intercept $\beta_{i0}$ represents intrinsic fitness, a species' fitness in the absence of interactions with neighbours. $N_{j}$ are the abundances of neighbours recorded for each observation, and the $\beta_{ij}$ represent the species-specific effect of each neighbour $j$ on $i$. Note that neighbours can include conspecifics, in which case intraspecific interactions are denoted as $\beta_{ii}$.
              
    \subsection{Unrealised links}
    
        \paragraph{}
        In any given site or year, a focal species may only be observed to interact with a subset of potential interaction partners, which means the IFM above will not be able to estimate all potential interactions ($\beta$) between species. This is especially true for rare species. 
        These \textit{unrealised} interactions can be estimated by using an alternative model with a different set of assumptions to the IFMs. This model is described as the response-effect model (REM) by Godoy, Kraft and Levine (2014) and assumes that each species has the same effect on all neighbours regardless of their identity, as well as the same response to competition regardless of competitor identity. 
        
        \begin{equation}
        F_{i} = \beta_{i0} - r_{i} \sum_{i}^{j} e_{j} N_{j}
        \label{rem1}
        \end{equation}
        
        Pairwise interactions which are missing from Eq. 1 can be approximated by Eq. 2 by multiplying the relevant $r_{i}$ and $e_{j}$ such that $\beta_{ij} = r_{i} e_{j}$. In order to allow both the IFM and REM interaction estimates to contribute to the likelihood, we first used the IFM to quantify observed interactions and then used those to estimate species-specific $r$ and $e$ parameters such that: 
    
        \begin{equation}
        r_i e_j \sim logistic \left ( \beta_{ij}, \sigma \right )
        \label{unrealised}
        \end{equation}
    
        where $\sigma$ is a community-level scale parameter for the logistic distribution. We use a logistic distribution here because the heavier tails make it a slightly weaker informative prior than a normal distribution. $\sigma$ defines how widely the tails extend and can be expressed in terms of the standard deviation. 

	\subsection{Model fitting}

        \paragraph{}        
        The IFMs (Eq. 1) and the REM (Eq. 2) can be implemented as generalised linear models in STAN \parencite{Carpenter2017}, a Bayesian statistical language where coefficient values will be estimated by MCMC sampling. The advantage of this approach is two-fold: the model can converge and coefficients can be estimated despite high model complexity and a large number of parameters. Using STAN requires translating the model formula into the STAN language, setting priors for parameters to be estimated, and using an indexing system to identify \textit{realised} interactions which are then fed into the REM. We provide a working example of the STAN file used to specify and set this model to the case study below in the Supplementary Information. From this file, only the link function for $F_i$ and it's parameterisation need to be modified in order to apply it to a different model or system. In the code given, a negative binomial distribution is used to fit seed production ($F_i$) but a different distribution may be more appropriate when using other proxies for lifetime reproductive success.   

        \paragraph{}
        STAN returns parameters as distributions which maximise the likelihood, and are conditioned by the data and priors. Priors describe the distribution of plausible values which these parameters may take. For an introduction to Bayesian inference which relates the use of priors to frequentist hypothesis testing, see \textcite{Ellison1996}. We recommend investigators experiment with setting different informed priors to both improve model convergence and verify the robustness of parameter estimates. The resulting parameters are termed posterior distributions, and samples from the posterior are drawn for analysis. \textcite{Ellison2004} also provides an accessible review of parameter estimate and the use of posterior distributions using a worked example on ant species richness data.

        %----------------------------------

    \section{Case study}

    	We applied the model framework described above to a case study of annual wildflower plants. After quantifying the interaction matrix, we scale all interaction samples using estimates of intrinsic fitness ($\beta_{i0}$) and experimentally-measured species demographic rates. The scaling is determined by a system-appropriate population dynamic model and allows us to translate the interaction matrix into per-capita effects on abundance. In the Results and Discussion, we illustrate potential applications of these \text

    	[Give justification for plant pop model]. 

    	\subsection{Data}

        \paragraph{}
        We applied this framework to annual wildflower community dataset from Western Australia. This system is a diverse and well-studied community of annual plants which germinate, grow, set seed and die within approximately 4 months every year. Individual fecundity data were collected in 2016, when 100 50 x 50 cm plots established in the understory of West Perenjori Reserve (29$^o$28'01.3"S 116$^o$12'21.6"E) were monitored over the length of the full field season, from July to October. The resulting dataset contains from 29 to over 1000 counts of individual plant seed production from 22 different focal species (with a median of 108 observations per species), in addition to the identity and abundances of all neighbouring individuals within the interaction neighbourhood of the focal plant. Interaction neighbourhoods varied in radius from 3 to 5 cm depending on the size of the focal species. Total neighbourhood diversity was 71 wildlfower species, 19 of which were recorded fewer than 10 times across the whole dataset. The species-specific effects of this latter group of species on focals were deemed negligible due to their extremely low abundance, they were thus grouped into an 'other' category and their effects on focals averaged. This resulted in 53 potential neighbour identities. Half of all plots were thinned (a quarter to 60\% diversity and a quarter to 30\%) to mitigate possible confounding effects between plot location and plant density, and did not target any particular species.

        \paragraph{} 
        We required species demographic rates (seed and germination) in order to scale model interaction estimates into per-capita interaction strengths. Species demographic rates for 16 of our focal species were estimated from a database of field experiments carried out between 2016 and 2019 where seedbags were placed in the field to estimate germination rates, and ungerminated seeds were evaluated in the lab for survivability. The remaining species were assigned mean demographic rates from these experiments. 

		\subsection{Model fitting}

        \paragraph{}
        We fit the model using R version 3.6.3, STAN and the rstan package \parencite{R2020, Carpenter2017, Rstan2020}. Estimates of seed production were fit with a negative binomial distribution. The model was run with 4 chains of 5000 iterations each, discarding the first 1000. Models were checked for convergence and traceplots were visually inspected to verify good chain behaviour and mixing. Model parameters were sampled 1000 times from the 80\% posterior confidence intervals to construct our parameter estimates. We then applied bootstrap sampling from each resulting interaction strength distribution to create 1000 samples of the community interaction network.

        \subsection{Scaling interaction estimates into per-capita effects}

        \paragraph{}
        The above model framework returns species-specific estimates of intrinsic fitness ($\beta_{i0}$), as well as as a species x neighbour matrix of observed ($\beta_{ij}$) and unobserved ($r_i e_j$) interaction estimates which quantify the effects of one neighbour $j$ on the intrinsic fitness of a focal species $i$. Though useful as they are, these estimates can lead to a wider range of potential applications when integrated into models of population dynamics. For example, we might be more interested in the effects of neighbours on the abundance or growth rate of a focal species rather than on it's proxy for lifetime reproductive success. Using an appropriate model of population dynamics, the raw interaction estimates can be scaled into per-capita effects (${\beta}''$) which are directly comparable between species \parencite{Godoy2014, Bimler2018}. % Note that this step depends on a population dynamic model which can link species abundance to the chosen measure of fitness, and may require species-specific measures of certain key demographic rates (e.g. mortality, seedling survival) to scale the interactions.

        \paragraph{}
		In order to determine the appropriate scaling for the interaction estimates returned by our framework, we transform the model into a form equivalent to a Lotka-Volterra competition model: 

        \begin{equation}
        F_{i} = \beta_{i0} \left ( 1 - \sum_{i}^{j} {\beta_{ij}}'' N_{j} \right )
        \label{LVform}
        \end{equation}

        This gives us the scaled interaction strengths: 
 
        \begin{equation}
        {\beta_{ij}}'' = \frac{\beta_{ij}}{\beta_{i0}}
        \label{scaling}
        \end{equation}

        \paragraph{}
		For this case study, we adapted the framework described here to work with a well-supported annual plant population model with a seed bank \parencite{Bimler2018}. The model describes changes in the rate of a focal species' seeds in the seed bank from one year to the next, which can be linked to $F_i$ through species-specific demographic rates (equations for the annual plant model are further described in Chapter 4). The exact form of the rescaled interactions can thus vary depending on the specific population dynamic model applied and may include other demographic rates which reflect species-level differences in growth and mortality.



\section{Results}

	\paragraph{} 
	We applied our framework to estimate non-trophic interactions between 22 focal species and 52 neighbours in an annual wildflower community. In our case study, 13.6\% of interactions between focal species were estimated using the REM (out of 484), and 40\% when taking all neighbouring species into account (out of 1144). IFM and REM estimates matched well and followed similar distributions (Figure \ref{fig:adist}). Interactions estimates were scaled into per-capita effects and the resulting interaction network is illustrated in Figure \ref{fig:all_vs_rem}.A.

	    \paragraph{} 
    One potential use for translating interaction estimates into per-capita interaction strengths is to identify species of particular importance to the healthy functioning of the whole community. Keystone species for example typically have low abundance but disproportionately strong effects on the rest of the community \parencite{Power1996}. Foundation species on the other hand are locally abundant and have important, sometimes facilitative effects on other species population dynamics \parencite{Ellison2019}. Figure \ref{fig:species}.A. identifies several contenders for keystone (orange diamonds) or foundation (blue diamonds) species status based on their community abundance and net interaction effects across all neighbours. Likewise, the integration of exotic species into native communities can also be evaluated using interaction networks. In our case study, two out of three exotic species (red diamonds in Fig. \ref{fig:species}) were extremely abundant but had substantially different effects on neighbours, with the grass \textit{Pentameris airoides} facilitating some species, whereas \textit{Arctotheca calendula} had almost exclusively competitive effects on all other neighbours (Fig. \ref{fig:species}.B.). These examples are not exhaustive, but serve to illustrate how interaction matrices can help us understand what roles specific species play within a community. 


    \begin{figure}[H]
        \hspace*{-3.5cm}
        \includegraphics[width=1.5\textwidth]{../2.analyses/figures/interaction_ests.png}
        \caption{Distribution of interaction estimates from our case study. Parameter estimates are sampled from the 80\% posterior confidence intervals returned by STAN. Upper left panel shows the distribution of observed interactions as estimated by the IFM ($\beta_{ij}$), which are then plotted against the corresponding REM estimates ($r_i e_j$) in the upper right panel. Bottom rows show the distribution of \textit{unobserved} (left) and observed interactions estimates returned by the REM. 
        . Interaction estimates are unscaled.}
        \label{fig:adist}
    \end{figure}

        \begin{figure}[H]
        \includegraphics[width=0.95\textwidth]{../2.analyses/figures/all_vs_rem.png}
        \caption{Mean per-capita interaction strengths between 22 focal species of wildflowers. (A) shows all observed and unobserved interactions whereas (B) shows those interactions estimated by the REM only. Competitive interactions are shown in yellow and facilitative in blue, line thickness denotes interaction strength. Arrows point to species $i$.}
        \label{fig:all_vs_rem}
    \end{figure}



        \paragraph{}
    Our resulting networks also differ qualitatively to association networks (Figure \ref{fig:netwks}.A-B, E), a common alternative to inferring interactions in species-rich communities. Firstly, intraspecific interactions ($\alpha_{ii}$) are estimated for all focal species and allow us to quantify how much a species regulates itself. In our case study, 85\% of all species across all network samples showed intraspecific competition. Interactions estimated on growth rates are also directed, which means they have an associated direction such as going from species $i$ to species $j$ ($\alpha_{ij}$), or alternatively from $j$ to $i$ ($\alpha_{ji}$). This interaction pair does not need to be of the same sign or strength, which makes the resulting interaction networks non-symmetrical. Interaction pairs of opposing signs were common in our case study (Figure \ref{fig:netwks}.D), making up 40\% of all pairs and many more differed in strength. 

    \paragraph{}
    In addition to explicitely allowing for many species interactions, including both competitive (Figure \ref{fig:netwks}.A) and facilitative (Figure \ref{fig:netwks}.B) interactions is another feature which distinguishes our framework from traditional approaches to plant community ecology. Plant population dynamic models typically force interactions to be competitive, despite rising evidence that facilitation is common and an important driver of plant community dynamics \parencite{Brooker2008a}. In our case study, 35\% of all focal x neighbour interactions were facilitative, as were 26 \% of all focal x focal interactions.



    \begin{figure}[H]
        \hspace*{-1cm}
        \includegraphics[width=1.2\textwidth]{../2.analyses/figures/spceffects_edited.png}
        \caption{Focal species net effects on neighbours, according to their abundance (A). On the x axis, values over 0 indicate a net competitive effect whereas values below 0 indicate a net facilitative effect on neighbouring species. Dashed lines represent the median value for focals. Diamonds are species means across all network samples, black lines cover the 50\% quantile and grey dots indicate the full range of out-strength values as calculated from 1000 sampled networks. Yellow, blue and red diamonds signify potential keystone or foundation species, and exotic species respectively. (B) Focal species effects on neighbours, split into competitive vs. facilitative. Axes refer to the absolute sum of competitive vs. facilitative interactions. (1) \textit{P. airoides}, (2) \textit{A. calendula}.}
        \label{fig:species}
    \end{figure} 


 %    Identifying keystone species. .. GITE and HAOD rel. abund < 0.01, two highest sumaji. Maybe GOBE  too? 
 %    Invasives...
 %     Foundation species: VERO, maybe POCA ? Also well connected! 
 %    PLDE : fairly common, strong facilitative effects
 %    Dominant species are locally abundant but replaceable. 
 %    Invasives: ARCA
 %    TRCY and TROR - very different strategies!
 %    EROD: outcompeted by many species, weak effects on others, received lots of facilitation
 % - abundance - sum of comp effects - sum of facil effects



    \newpage

    \begin{figure}[H]
        \includegraphics[width=\textwidth]{../2.analyses/figures/bigntwkfig.png}
        \caption{Competitive (A) and facilitative (B) per-capita interaction networks estimated from our model framework, compared to an association network (C) estimated from the same data using the cooccur package \parencite{Griffith2016}. Competitive and facilitative interactions are here shown separately for ease of view but were analysed together (see Fig. \ref{fig:all_vs_rem}.A). Focal species only are included. Species associations (C) were all negative.  Interactions estimated for (A) and (B) are given as the mean over 1000 samples. Line thickness denotes the strength of the interactions or association. (D) shows the proportion of pair-wise interaction loops which were asymmetric (+/-), cooperative (+/+) or competitive (-/-) in our interaction matrix.}
        \label{fig:netwks}
    \end{figure}    



\section{Discussion}
    
    \paragraph{}
    The framework presented here allows the estimation of species-rich interaction matrices from plant community data. Given individual observations of reproductive success and the identity of the local neighbourhood, our model uses Bayesian regression to quantify the competitive or facilitative effects of one species on another's intrinsic fitness. Scaling the resulting interaction estimates into per-capita interaction strengths can integrate these results into  population dynamic models, extending traditional plant ecology approaches to quantifying interactions by explicitly allowing for many species and interactions, as well as being able to include and estimate facilitation. We showcase this framework with a case study on a diverse annual wildflower system and illustrate how the resulting interaction networks allow us to better characterise the inner workings of these communities and has strong applications for ecosystem management and conservation.
 

    \paragraph{}
    The model framework we present here quantifies pair-wise interaction strengths between species from diverse single-trophic community data. Several aspects of our model and its implementation allow us to better capture complex features of plant communities compared to co-occurrence networks as well as traditional plant population dynamic models. Like co-occurrence networks, our framework is flexible to the inclusion of many species. Unlike co-occurrence networks however, interactions in our model are estimated from species effects on each other's growth rates. This allows us to capture how interactions may directly affect species abundances, rather than spatial association patterns. Two features make this possible: the response-effect model to estimate unobserved interactions, and a Bayesian implementation to deal with many parameters. 

    \paragraph{}
    The response-effect model allows us to deal with a small number of unobserved interactions by relaxing assumptions for those cases and estimating more general 'effect' and 'response' parameters from the pair-wise interactions that are observed. Because they are estimated from observed interactions previously quantified by the IFM, the response and effect parameters are constrained such that their product follows a similar distribution to that of the observed interactions (Figure \ref{fig:adist}). 


    \paragraph{}
    The Bayesian implementation means that parameters are estimated using MCMC algorithms, which amongst many benefits can estimate more parameters than with more commonly used frequentist approaches \parencite{Dorazio2016}. This approach also returns parameter estimates as posterior distributions rather than simple mean and standard deviations. By having access to the full distribution of likely values for every interaction, we can draw samples of likely networks from these distributions and thus easily incorporate some measure of variation and uncertainty around interaction values. 

    \subsection{Limitations}
        
        \paragraph{}
        Though our framework does provide a way of estimating interactions between species which are not observed to cooccur, care should be taken that these unrealised interactions do not dominate a resulting network. The method we propose uses observed interactions to estimate simpler parameters which allow us to approximate the remaining interactions. This method thus works better the more observed interactions we have with which to estimate the unobserved ones, and the more reliable those observed estimates are. It is also important to consider the likely reasons for missing interactions. Forbidden links are a subset of potential interactions which cannot be observed, often due to biological constraints or spatio-temporal uncoupling. For example, a pair of short-lived annual plants might have such opposing phenologies that their growing season never overlaps in the field. This type of interaction would be assigned a 0 value. We direct the reader towards the literature on forbidden interactions \parencite{Olesen2011, Jordano2016} for solving these cases. 

        \paragraph{}
        Our model framework is aimed to helping empiricists who would like to estimate species interactions in non-trophic communities. Though the MCMC sampling algorithm does allow for many parameters to be estimated, it is crucial to check chain-mixing and model behaviour to verify that the full parameter space is sampled. The amount of data required for sampling diverse communities may still be substantial as we do not recommend fewer than 20 observations per focal species. Grouping species which appear very rarely is also a common strategy to avoid over-parameterisation. In our case study, neighbour species which were recorded fewer than 10 times in each habitat type were grouped into a 'rares' category with its own interaction effect. If interactions with or between rare species are the object of interest however, data-collection can simply be focused on amassing observations of focal individuals from those rarer species to be able to estimate these interaction strengths.   


    \subsection{Applications to ecosystem management}

        \paragraph{}
        Interaction networks provide a unique lens with which to examine an ecological community. By describing how species are linked, networks can for example be used to evaluate ecosystem response to human-altered landscapes and guide future management decisions \parencite{Ross2011} or explore how communities may respond to global warming \parencite{Gorman2019}. Specifically, couching species-rich interaction matrices into models of population dynamics allow us to determine how community dynamics may affect and respond to individual species. This can direct ecosystem management efforts such as prioritising certain species to conserve, or eradicating invasives which are disruptful to native species community dynamics.
        % Because some species have stronger effects on other species abundances and are more crucial to the maintenance of biodiversity, management efforts can be prioritised towards their conservation. 

        \paragraph{}
        Certain species play important roles in the community which can be deduced from their position in the interaction network \parencite{Cirtwill2018a}. Keystone species for example are both crucial to maintaining the organisation and diversity of their community, and have exceptionally strong effects on other species \parencite{Mills1993}, often leading to secondary extinctions when they are removed from a network. Their influence on the abundance of other species and wider community dynamics is disproportionately high compared to their biomass \parencite{Power1996, Piraino2002, Libralato2006}, which allows easy identification of potential keystone species from our network results (Fig. \ref{fig:species}) though further tests are recommended for validation. [The two species were \textit{Gilberta tenuifolia} and \textit{Haloragis odontocarpa} in case that's interesting.] Foundation species also control local biodiversity but are widely connected, locally abundant and hold a unique position in regards to their effects on community dynamics \parencite{Ellison2005, Baiser2013, Ellison2019}. Though it can take many years of careful study to identify foundation species from abundant species which are not as crucial to community structure, our framework can help distinguish species which are abundant and interact weakly with neighbours from common species which have strong competitive and facilitative effects on the community. Identifying keystone, foundation and other important types of species roles is helpful for maintaining biological diversity, ecosystem integrity and functioning, especially in response to disturbances and other stresses \parencite{Nyakatya2008, Orwin2016, Losapio2017, Narwani2019}.

        \paragraph{}
        Network approaches can also be used to evaluate how specific species are integrated within the community. Our case study for example illustrates how different exotic species can have widely varying competitive and facilitative effects on neighbours (Fig. \ref{fig:species}). Though many invasive species compete with natives \parencite{Naeem2000, Riley2008, Zheng2015}, several studies have found evidence of invasives facilitating natives, with cascading effects on other species and net positive effects on ecosystem processes \parencite{Rodriguez2006, Ramus2017}. Alternatively, threatened species could also benefit from this type of analysis. Recent work suggests rare species are the prime emitters and beneficiaries of facilitative interactions, forming mutualistic refuges from competitive dominants \parencite{Calatayud2019, Hines2020}. Identifying neighbouring species which facilitate and increase the local abundance of rarer, endangered target species is one of many strategies which could help guide conservation management.


    \subsection{Understanding community diversity and stability}

        \paragraph{}
        Quantifying the matrix of per-capita interaction strengths between species in horizontal communities allows us to explore how the mechanisms maintaining diversity and stability operate in these systems and across a broad number of species. Self-regulation, for example, is an extremely important driver of community stability \parencite{Barabas2017} and arises from how individuals of the same species interact with one another. Whereas same-species effects cannot be estimated by association networks, our framework quantifies intraspecific interactions and allows us to measure the strength and prevalence of competitive and facilitative density-dependence. Measures of intra and interspecific interactions can also allow us to estimate niche overlap between species (for an example, see \textcite{Chu2015a}): weak interactions between species indicate they are not sharing or competing for many resources, and occupy different niche spaces in the community (ref??). Strong competitive self-regulation, and weak interactions between two species, are two of many mechanisms leading to stability and diversity in plant communities.

        \paragraph{}
        A key feature of our model framework is the inclusion of facilitative interactions, which have long been disregarded in plant population models and theoretical frameworks of plant diversity-maintenance. The importance of facilitative interactions to community structure and patterns of abundance has long been recognised \parencite{Callaway1997a} but there is still little consensus on how they may affect biodiversity \parencite{Bruno2003}. Recent work suggests facilitation may be more widespread than expected \parencite{Gross2015, Picoche2020} and can  benefit species diversity and stability depending on the circumstances \parencite{Coyte2015, Brooker2008}. Our framework provides a means to investigate the prevalence and strength of facilitation across multiple species, and how it may act in relation to competition and species diversity.  
        
        \paragraph{}
        Ultimately, quantifying plant interaction networks allows us to apply tools from network theory which will help us understand not only how plants interact, but also how these interactions drive community-level patterns of abundance and diversity. Several metrics already exist for describing network structure such as weighted connectance \parencite{Ulanowicz1991} or relative intransitivity \parencite{Laird2006a}, though these are fewer than for trophic or unweighted networks networks (e.g. \textcite{Bersier2002, Delmas2019}. Adapting measures of nestedness or modularity for example to non-sparse networks (as plant communities typically are) would allow us to further characterise how interactions and species are organised. These metrics relate to various aspects of stability and could greatly inform us on how diversity is maintained between plants. 

        %Likewise, networks also provide several ways of measuring and describing species roles in their respective communities \parencite{Cirtwill2018a} for example through the use of structural motifs, unique patterns of interacting species which together make up the whole network. Motifs have been found to have important biological meaning in food webs \parencite{Bascompte2005a} but remain to be identified for single-trophic and bipartite networks. 



    % \subsection{Future research directions}

    % - adding trait or phylogeny data, for example by informing priors with trait or phylogeny distance matrix
    % ie. improving our estimates \\
    % - improving our knowledge of networks, developing tools for plant networks and tools that can deal with non-sparse networks, competitive and facilitative interactions \\
    % - integrate with trophic networks (which appear to be sensitive to producer/plant network structure!)\\


% \bibliography{../../../BibTex_files/03_Chapter3}
% \bibliographystyle{plainnat}
\printbibliography    

\section{Supplementary Information}

    \subsection{STAN model code}

    \includepdf[pages=-]{../1.model/model_stan.pdf}

\end{document}