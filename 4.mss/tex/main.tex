% DOCUMENT CLASS
    \documentclass[a4,12pt]{article}

	
    \usepackage{authblk} % For multiple authors
    \usepackage{amsmath}
    \usepackage[utf8]{inputenc}
 	\usepackage{times}
 	\usepackage{setspace} % lines spacing
    \usepackage{lineno} 	% See \linenumbers at the top of content.tex
    \usepackage{booktabs}
    \usepackage[skip=2.5\baselineskip]{caption} % to give more space between figs and captions
    \usepackage{listings} % allows me to insert code 
    \usepackage{pdfpages}
    \usepackage[margin=20mm]{geometry} % set exact margins 
    \usepackage{caption} % allows doublespacing of captions
    \usepackage{hyperref} % allows adding urls
    \captionsetup{font=doublespacing}

% REFERENCES
    \usepackage[firstinits=true, backend=biber, style=authoryear, maxcitenames=2, url=false, isbn=false, doi=true, eprint=false]{biblatex}
    \DeclareNameAlias{sortname}{last-first}
    \DeclareSourcemap{				% this prints url for webpage references
 	 \maps[datatype=bibtex, overwrite=true]{
  	  \map{
   	   \step[fieldsource=url, final]
   	   \step[typesource=misc, typetarget=online]
    }
  }
}
    \addbibresource{../../../../BibTex_files/03_Chapter3.bib}

% GRAPHICS
    \usepackage{graphicx} % More advanced Fig. inclusion
    \usepackage{float} % For specifying table/Fig. locations, i.e. [ht!]
    \usepackage{changepage}
    \usepackage[table,xcdraw]{xcolor}

\doublespacing 

\title{\large Estimating interaction strengths for diverse horizontal systems using performance data.}



\author[1]{\small Malyon D. Bimler *}
\author[2]{\small Margaret M. Mayfield}
\author[3]{\small Trace E. Martyn}
\author[4]{\small Daniel B. Stouffer}

\affil[1]{\footnotesize School of BioSciences, The University of Melbourne, Parkville, Victoria, Australia. Email: malyon.bimler@unimelb.edu.au}
\affil[2]{\footnotesize School of BioSciences, The University of Melbourne, Parkville, Victoria, Australia. Email: margie.mayfield@unimelb.edu.au}
\affil[3]{\footnotesize Yale School of the Environment, New Haven, Connecticut, USA. Email: martyn.ecology@gmail.com}
\affil[4]{\footnotesize Centre for Integrative Ecology, School of Biological Sciences, University of Canterbury, Christchurch, New Zealand. Email: daniel.stouffer@canterbury.ac.nz}



\setlength{\intextsep}{10ex}

\begin{document}
\maketitle  

% \noindent
% \textbf{Running title:} Estimating horizontal interaction matrices

\noindent
\textbf{Corresponding Author:} Malyon D. Bimler, email: malyonbimler@gmail.com \\ %, ph: +64 481941634\\


\noindent
Number of words - Abstract: 225\\
Number of words - main text: 5942\\
3 Figures. in main text.\\
Number of references: 67\\


\noindent
\textbf{Keywords:} Annual plants, competition, facilitation, interaction strength, network, non-trophic, population dynamics, species interactions.  

\section*{Data accessibility}

Code is available in the GitHub repository \url{https://github.com/malbion/JointModelFramework}. Data presented in the case study is available from the Dryad  Digital Repository: \url{https://doi.org/10.5061/dryad.h44j0zpq3}. 

\newpage


% \linenumbers

\section*{Abstract}
    
    \begin{enumerate}
    \item{Network theory allows us to understand complex systems by evaluating how their constituent elements interact with one another. Such networks are built from matrices which describe the effect of each element on all others. Quantifying the strength of these interactions from empirical data can be difficult, however, because the number of potential interactions increases non-linearly as more elements are included in the system, and not all interactions may be empirically observable when some elements are rare.}
    \item{We present a novel modelling framework which uses measures of species performance in the presence of varying densities of their potential interaction partners to estimate the strength of pairwise interactions in diverse horizontal systems.}
    \item{Our method allows us to directly estimate pairwise effects when they are statistically identifiable and to approximate pairwise effects when they would otherwise be statistically unidentifiable. The resulting interaction matrices can include positive and negative effects, the effect of a species on itself, and allows for non-symmetrical interactions.}
    \item{We show how to link the parameters inferred by our framework to a population dynamics model to make inferences about the effect of interactions on community dynamics and diversity.}
    \item{The advantages of these features are illustrated with a case study on an annual wildflower community of 22 focal and 52 neighbouring species, and a discussion of potential applications of this framework extending well beyond plant community ecology.}
\end{enumerate}

\begin{refsection}

\section{Introduction}

    
    \paragraph{}
    In many biological systems, interactions between system elements (be these species, individuals, etc.) affect population-level performance and together determine the dynamics of the whole system. To understand system dynamics when multiple system elements are involved, complex systems can be represented as networks where the elements are nodes and linked by interactions \parencite{Pimm1978}. These nodes can take on a wide array of identities, including cells, individuals, populations or species. Likewise, interactions or links can operate via many different mechanisms and have a wide range of effects on the nodes. Network theory has been widely applied to investigate different biological systems. It has been particularly effective at informing us of the underlying biological processes structuring diverse multi-trophic communities \parencite{Dunne2002, Thompson2012} through the study of vertical interactions in food webs, plant-pollinator and host-parasite systems \parencite{Lafferty2008, Stouffer2014, Cirtwill2015a}.

    \paragraph{} 
    Horizontal networks, however, where interactions occur within the same level of organisation (for example interactions between plants belonging to the same food web) \parencite{Vellend2016} have been more neglected by network ecology \parencite{Ellison2019}. In such systems, interactions between species are not always easy to directly observe empirically and must instead be deduced through other means. A common approach in population ecology is to directly quantify the effects of interactions on a species of interest by evaluating performance in the absence and presence of potential interaction partners at fixed or varying densities \parencite{Connell1961, Grace1990}. `Performance' here refers to any variable that affects the dynamics of the system, for example quantity of resources gathered, biomass accumulation, or population growth rate. Measuring species interactions as effects on performance allows us to infer future population trajectories through their connection to population dynamics models, and thus helps draw direct conclusions about the effects of interactions on emergent diversity patterns \parencite{Laska1998}. The resulting interactions are phenomenological and thus not dependent on any specific mechanism, allowing us to capture a wide range of biological processes affecting the dynamics of the whole system \parencite{Novak2010}. Such methods can quickly become data intensive and computationally complex, however, as the number of species $S$ increases and the number of potential direct interactions subsequently increases as $S^2$. Highly diverse systems pose a further challenge: the abundance distribution of different species is typically skewed, with a few species making up the majority of abundances and a large number of elements remaining rare \parencite{Fisher1943}. Given that data collection is limited in time and scope, interactions with rarer species may not be observed simply by chance. We thus run the risk of excluding them from analyses regardless of the role they might play \parencite{Olesen2011}. Empirically quantifying interaction matrices for diverse horizontal systems thus requires a method that is flexible to both a high number of species, and potential gaps in our records of interactions. 

    \paragraph{}
    Various methods have been developed to circumvent these issues. A common strategy is to reduce the number of parameters to be estimated by assuming most interactions are weak enough to be negligible, and thus priority is given to inference of only the strongest interactions \parencite{Weiss-Lehman2022}. Other approaches include averaging interactions across species by aggregating species together in groups, for example based on their origin and life form \parencite{Martyn2021}, by their taxonomy and traits \parencite{Uriarte2004}, or lumping all heterospecific species together \parencite{Chu2015}. Here we present a alternative approach designed to make the most of available data without requiring such strong a priori hypotheses. Specifically, we develop a joint model that allows us to estimate both identifiable and unidentifiable pairwise interactions from measures of performance in the absence and presence of different interaction partners.

    \paragraph{} 
    We present a general framework to estimate interactions in diverse horizontal systems. We implement the model in STAN \parencite{Carpenter2017}, a Bayesian statistical language, and apply it to an ecological case study of an annual wildflower community in Western Australia. Using this dataset, we estimate positive and negative interactions between 22 focal species and 52 neighbouring species and illustrate a range of uses for this approach through ecologically-relevant findings. We further describe how to couch the returned interaction estimates into established models of population dynamics, thus allowing inferences to be drawn between the structure and nature of the interaction matrix and patterns of community abundances and biodiversity. This framework presents a new and exciting way to make use of data that would otherwise be too incomplete to uniquely infer all pairwise interactions. We also provide model code in R and STAN which requires little to no modification for immediate application to a wide variety of datasets of species performance.


\section{Methods}

\paragraph{} 
We developed a joint modelling framework to estimate pairwise interactions which benefits from several distinguishing features including the ability to estimate both \textit{identifiable} interactions (direct estimates from the observed data) and \textit{unidentifiable} interactions (when observations are missing or too few). After describing the required data (\ref{meth:data}), we show how one can estimate identifiable interactions with a unique interaction parameter as described in the neighbour-density dependent model (\ref{meth:nddm}). We then define and select which interactions are identifiable and which are not (\ref{meth:id_params}) based on data availability. Independent from the first model, we also describe a response--impact model (\ref{meth:rim}) where species have a singular effect on neighbours and a singular response. This allows us to estimate unidentifiable interactions as the product of element-specific response and impact parameters. Both models contribute to the overall joint model likelihood as detailed in \ref{meth:addlog}. Together, identifiable and unidentifiable interaction estimates can then populate community interaction matrices that describe the effects of all interaction partners on the performance of all focal species. 


    \subsection{Data requirements}
    \label{meth:data}

    \paragraph{}
    The joint model framework was initially developed for an ecological dataset where interacting elements (species) affect each other's performance (lifetime reproductive success). Though we refer to system elements as species throughout this paper, this framework can be applied to data from any interacting group of elements (e.g.\ cells, individuals, populations, species) which meet the following criteria. First, observations must include some proxy for performance, such as growth (e.g.\ biomass), fecundity (e.g.\ number of eggs laid) or chemical production (e.g.\ oxygen). Second, these observations must also record the identities and densities of elements which potentially interact with each focal element. Lastly, observations should be replicated for each focal element with the aim of capturing variation in the identities and densities of interaction partners. Though not strictly necessary, it is also beneficial to have observations of focal elements with no interaction partners to better estimate intrinsic performance. 

    \paragraph{}
    We define $s$ as the number of focal species $i$, $i \in \{ 1, .., s \}$, and $t$ as the number of interacting species $j$ across all $s$ focals, $j \in \{1, ..., t \}$. Typically, not all species in the system will be represented in the set of focals, such that $s \le t$. Measurements of the performance of individual units from each focal element (e.g.\ seed production of individual plants belonging to a set of focal species) are stored in a vector $p$ of length $n$ and indexed by $k$, with $k \in \{1, ..., n \}$. The densities of interaction partners are stored in a matrix $X$, of size $n \times t$. When an element $j$ was absent for a given observation $p_k$, then $X_{k,j} = 0$. Finally, the species identity of each of the $n$ focal individuals is stored in the vector $d$, of length $n$, and containing the index of the corresponding species: $d_k \in \{1, ..., s \}$.

    \subsection{Neighbour-density dependent model}
    \label{meth:nddm}    

        \paragraph{}
        We quantify the strengths of interactions by regressing the performance of a species (alternatively, a population or any chosen set of replicated units) against the density and identity of other interacting species in a neighbour-density dependent model (NDDM). Increases or decreases in a species' performance are thus attributed to the changing densities of its interaction partners. 

        \paragraph{}
        We implement a NDDM for each focal species $i$ which regresses the densities of species $j$ ($j = 1, ..., t$) against the measured proxy for performance $p_{k}$ through a link function $f(p_k)$:
        \begin{equation}
        f(p_{k}) = \gamma_{d_k} - \sum_{j=1}^{t} \beta_{d_k,j} \, X_{d_k,j}
        \label{nddm}
        \end{equation}
        The parameters $\beta_{d_k,j}$ capture the effect of each species $j$ on $i$ whereas the intercept $\gamma_{d_k}$ represents intrinsic performance (in the link scale), a species' performance in the absence of interactions or when all interaction effects are $0$. Note that interacting species $j$ can include members of focal species $i$ itself, in which case intraspecific interactions are captured by the parameter $\beta_{d_k,i}$. Furthermore, the equation above places no restrictions on the sign of $\beta_{d_k, j}$: interactions can be harmful to the focal species (competitive) or beneficial (facilitative). The negative sign in front of $\beta$ does, however, mean that positive interaction estimates should be interpreted as competitive and negative estimates as facilitative. Note that because $d_k$ is the focal species index for each observation, parameters making use of the $d_k$ subscript can use an $i$ subscript instead when they are no longer linked to a specific observation of performance.

        \paragraph{}
        We implement this model, as well as the RIM described below (in \ref{meth:rim}), as generalised linear models. This means that the relationship between $p_k$ and the right-hand side of the equation does not need to be linear. This can easily be changed by choosing an appropriate link function $f(p_k)$ for the data in question. In our case study, we use the link function $f(p_k) = \ln(p_k)$ to model our response variable as a negative-binomial variate.
              

    \subsection{Defining identifiable and unidentifiable interaction parameters}
    \label{meth:id_params}

    \paragraph{}
    A common issue in observational datasets is that some species or elements are not observed interacting with other species or interacting at sufficiently variable densities because data sampling is limited. This can create a situation in which we cannot estimate all potential interaction parameters ($\beta_{i,j}$) in the NDDM specified above, especially for rare species $j$. For an interaction parameter to be \textit{identifiable}, and thus inferrable by the NDDM, data must contain measurements of the performance of $i$ when interacting with $j$ at varying densities. Moreover, the vector of densities of $j$ associated with measurements of focal $i$ must be linearly independent of all other vectors of densities of species interacting with $i$. For example, if two species $a$ and $b$ interact with focal $i$ yet have the same density or equally proportional densities at every measurement, neither $\beta_{i, a}$ nor $\beta_{i, b}$ is inferrable by the NDDM. We define \textit{identifiable} interactions as those which are inferrable following the above assumptions, and \textit{unidentifiable} interactions as those which are not. Given an empirical dataset, we construct a matrix $Q$ of size $s \times t$, with $Q_{i, j} = 1$ if the corresponding $\beta_{i, j}$ parameter is identifiable, and $Q_{i, j} = 0$ if not. We describe our verification of linear independence between vectors of neighbour densities and the construction of the $Q$ matrix in the Supplementary Methods \ref{SI:identify} and in the GitHub repository.
    
    \subsection{Unidentifiable parameters and the response--impact model}
    \label{meth:rim}

    \paragraph{}
    Having unidentifiable parameters creates multiple problems which we would like to overcome. For example, we might wish to make out-of-sample predictions about the consequences of interactions between two species $i$ and $j$ that were never observed to interact but could still potentially interact under some environmental conditions; without knowing the corresponding $\beta_{i,j}$ and $\beta_{j,i}$, this is impossible. When an interaction coefficient is unidentifiable because of insufficiently variable neighbour densities, we still want to allow such neighbours to have non-negligible impacts on focal plants---as opposed to assuming $\beta_{i,j} = 0$ or dropping that density predictor altogether.
    Approximating unidentifiable interaction parameters is a complex challenge with multiple potential solutions. Here, we address the issue by using a model which assumes that a species $i$ will typically have a singular impact ($e_i$) on and a singular response ($r_i$) to neighbours independent of neighbour identity (response and impact model, or RIM - see \cite{Godoy2014b}). A pairwise interaction parameter is therefore the product of a focal species $i$'s' response parameter and an interacting species $j$'s' impact parameter. The corresponding density-dependent model of performance then becomes:
        \begin{equation}
        f(p_{k}) = \gamma_{d_k} - r_{d_k} \sum_{j=1}^{t} e_{j} X_{d_k, j}
        \label{rim}
        \end{equation}
    To fit the RIM, the densities of species $j$ must be linearly independent to the densities of other species and/or combinations of species, across the entire $X$ matrix. This is in contrast to \ref{meth:nddm} above, where the densities of $j$ must be linearly independent within the subset of observations for each focal $i$. Importantly, this is a less strict condition for parameter identifiability. As a result, it is frequently possible to estimate pairwise interactions that would be unidentifiable given Eq.~\ref{nddm} by recognising that pairwise density dependence in Eq.~\ref{rim} is given by $\beta_{i, j} = r_{i} \, e_{j}$.

    Readers should note that the response--impact model makes different assumptions about the structure of interactions in nature. Specifically it supposes that species tend to have a generalisable effect on and response to neighbours regardless of neighbour identity. This stands in contrast to the NDDM, which assumes a unique parameter for every interaction thereby allowing for more idiosyncrasies across species. Such simplification is useful from a statistical point of view in order to infer unidentifiable interactions, but its ecological significance should not be dismissed. Though there is evidence in support of this approach \parencite{Stouffer2022, Skwara2022}, care should always be taken to assess the different assumptions behind the estimation of identifiable and unidentifiable interactions and how well these may apply to any particular system of interest.

    \subsection{Joining the two models}
    \label{meth:addlog}

    \paragraph{}
    The interaction parameters returned by Eqs.~\ref{nddm} and \ref{rim} can be used to construct a community interaction matrix $B$ of size $s \times t$, where the effect of species $j$ on focal species $i$ corresponds to value in the $i$'th row and $j$'th column of $B$. An approximation of the matrix of interactions $B$ is thus given by:
        \begin{equation}
        B = Q \circ \tilde B + (1 - Q) \circ r e^T
        \label{matB}
        \end{equation}
    where $Q$ is the matrix of identifiable interactions, and $\tilde B$ is an $s \times t$ matrix that only has free parameters inferred in the locations where $Q_{i, j} =1$. The symbol $\circ$ represents the element-wise (Hadamard) product, whereby each element $i, j$ of the first matrix is multiplied by the $i, j$ element of the second matrix, resulting in a matrix of the same dimension as the operands. In other words, the final interaction matrix $B$ takes a free parameter from the NDDM when that parameter is identifiable, and the corresponding RIM estimate, $r_i e_j$ when that parameter is not.

    \paragraph{}
    An important novel feature of our framework is that we can fit all parameters simultaneously by requiring that both the NDDM and RIM contribute to an overall joint model log likelihood:
         \begin{equation}
        \mathcal{L}_{joint} = \mathcal{L}_{RIM} + \mathcal{L}_{NDDM}
        \label{loglikjoint}
        \end{equation}
    with
        \begin{equation}
        \mathcal{L}_{RIM} = \sum_{k=1}^{n} \ln ({\rm Pr} (p_k | \gamma, r, e, \dots))
        \label{logrim}
        \end{equation}
        \begin{equation}
        \mathcal{L}_{NDDM} = \sum_{k=1}^{n} \ln ({\rm Pr} (p_k | \gamma, B, \dots))
        \label{lognddm}
        \end{equation}
    where both sums are across all $n$ observations, ${\rm Pr}$ is the probability density for the model (e.g.\ negative binomial), $p$ is the vector of measured performances, $\gamma, r, e,$ and $B$ are parameters as defined in Eqs.~\ref{nddm} and \ref{rim}, and $\dots$ includes any other parameters (e.g.\ dispersion if using a negative binomial distribution for the data). Note that $p_k$ and $\gamma$ are common to both the RIM and NDDM.

    \paragraph{}
    For a hypothetical dataset with which all interactions are identifiable, then $B = \tilde B$ and fitting the RIM is not strictly necessary. The parameter vectors $r$ and $e$ would still be estimated, but they are independent of the parameters in $\tilde B$ and hence maximisation of both likelihoods is independent. Conversely, if no interactions are identifiable then the joint model devolves to the RIM only. In the middle ground---where the majority of datasets are likely to lie---the joint model framework allows us to estimate free interaction parameters when possible, and use $r_i e_j$ estimates when not. An important distinction to make is that the NDDM estimates identifiable interactions only, whereas the RIM estimates all interactions, pairwise identifiable and pairwise unidentifiable. However, by maximising Eq. \ref{loglikjoint} we allow both models to provide good fits to the data but also for $r$ and $e$ to ``adjust'' around inferred values of $\tilde B$. 


    \subsection{Making interaction parameters comparable across focal species} 

        The $B_{i, j}$ estimates returned by the framework in the composite matrix $B$ describe the effect of species $j$ on the performance of species $i$. Differences in the magnitude of these interaction terms may thus reflect intrinsic differences in performance, which can vary in time, space and between species. Two species may for example have different baseline values of reproductive fitness. In order to make the effects of interactions comparable between species, the interaction terms returned by both models above can be transformed into scaled interaction strengths \parencite{Laska1998}. The appropriate scaling is determined by rewriting the neighbour density-dependent model (Eq. \ref{nddm}) into a form equivalent to a Lotka-Volterra competition model: 
        \begin{equation}
        f(p_{k}) = \gamma_{d_k} \left ( 1 - \sum_{j=1}^{t} {\beta}''_{d_k, j} X_{d_k, j} \right )
        \label{LVform}
        \end{equation}
        This reveals that our interaction terms can be rescaled into interaction strengths by dividing them by the recipient element's intrinsic performance:  
        \begin{equation}
        {\beta}''_{i, j} = \frac{B_{i, j}}{\gamma_{i}}
        \label{scaling}
        \end{equation}
        Though this scaling step is not strictly necessary, for ecological datasets these scaled interaction strengths have the benefit of being directly comparable both across species and across environmental contexts where reproductive fitness is likely to vary \parencite{Wootton2005}.

    \subsection{Integrating interaction strengths into models of population dynamics}

        \paragraph{}
        In certain instances, models of species population dynamics can be used to extend the usefulness of the framework presented here. We suggest two cases where this may be useful. Firstly, the variable chosen to measure the performance of focal species $p$ may not directly translate into a measure of performance which is relevant to system dynamics, due to practical constraints with collecting empirical data. For example, the life-history reproductive strategies of certain species may lead to measures of high performance in the field which do not account for low survival rates post-observation \parencite{Broekman2020}. In these cases, population dynamics models can be used to include additional species-specific demographic rates into estimates of interaction effects. Alternatively, we might be more interested in the effects of interacting species on the density or growth rate of a focal species rather than on its performance. In this scenario, a population dynamic model can be used to translate interaction effects on performance into interaction strengths affecting the variable of interest. 

        \paragraph{}
        In both cases, an established population dynamic model is required as well as knowledge of crucial species-specific demographic rates. This step is illustrated for our case study in the Supplementary Methods \ref{SI:popdyn}, where we use annual plant population dynamic model to transform effects on wildflower seed production (proxy for performance) into effects on population growth, and includes species-specific estimates of seed germination and survival rates. 


    \subsection{Model fitting}

        \paragraph{}        
        We implement the NDDM (Eq.~\ref{nddm}), the RIM (Eq.~\ref{rim}), and the joint model as generalised linear models in STAN \parencite{Carpenter2017}, using R version 3.6.3 \parencite{R2020} and the rstan package \parencite{Rstan2020}. Using STAN requires translating the model formula into the STAN language, setting priors for parameters to be estimated, and using an indexing system to discriminate between identifiable and unidentifiable interactions. We provide a working example of the STAN code as well as R scripts and functions to run and fit the model on a simulated dataset in our GitHub repository \url{https://github.com/malbion/JointModelFramework}, see Supplementary Methods \ref{SI:modelcode} for additional comments on the code.
        From the model file, only the family function, the link function for $p_k$ and its parameterisation need to be modified in order to apply it to a differently-distributed dataset. Additionally, non-integer measures of performance (e.g.\ biomass) should be redefined as real rather than integers in the data block. In the code given, a negative binomial distribution is used to fit seed production, but a different distribution may be more appropriate when using other measures of performance.   
        \paragraph{}
        STAN returns parameters as distributions which maximise the likelihood, and are conditioned by the data and priors. Priors describe the distribution of plausible values which parameters may take. For an introduction to Bayesian inference which relates the use of priors to frequentist hypothesis testing, see \textcite{Ellison1996}. We recommend investigators experiment with setting different informed priors to both improve model convergence and verify the robustness of parameter estimates. The resulting parameters are termed posterior distributions, from which samples are drawn for analysis. Using parameter distributions rather than point estimates allows for easy inclusion of uncertainty, we therefore recommend sampling from each posterior interaction strength distribution to create multiple samples of the community interaction matrix. See \textcite{Ellison2004} for a comprehensive review of parameter estimations.


    \subsection{Assessing model convergence}

        \paragraph{}
        We first evaluated convergence of the NDD-only model, the RI-only model, and the joint NDD-RI model when fit to simulated data created with the \texttt{simul\_data()} function available in this project's associated GitHub repository. In these test fits, we ran the models with 4 chains and 3000 iterations, of which the first 2000 were discarded. We varied the total number of species and the proportion of unidentifiable interactions across our simulated datasets; when fitting the NDDM-only model, unidentifiable interactions were assigned a value $\beta_{i,j} = 0$. We observed good convergence of the $\gamma$ and $B$ parameters in all models as evaluated by the $\hat{R}$ statistic ($\hat{R} < 1.01$) and visual inspection of traceplots (results not shown here).
        As expected for the RI-only model and the joint model, our latent variables $r$ and $e$ often showed sign switching. This means that different MCMC chains returned coefficient values which are of the same magnitude but with opposing signs. Whilst this affects the $\hat{R}$ statistic of these $r$ and $e$ parameters, it does not affect the convergence of the resulting interaction parameters (i.e.\ $r_i e_j$), hence why we excluded $r$ and $e$ from our evaluation of convergence.

    \subsection{Dealing with sparse networks}
    \label{meth:sparse}

        \paragraph{}
        In horizontal systems, interaction networks are expected to be non-sparse because species typically interact via a small set of shared, limiting resources. Nonetheless, it is likely that the posterior distributions of some interaction estimates returned by the model will overlap with zero. An overlap with zero may be due to several factors and does not necessarily equate to an interaction being insignificant. Firstly, an overlap with zero may arise when an interaction is positive or negative depending on local conditions. Secondly, it may indicate the interaction is poorly informed and hence overlaps with zero because it has a large posterior distribution reflect a lack of confidence to its effect. Lastly, the interaction may be well-informed but weak, in which case the posterior is centred around zero.

        \paragraph{}
        If, contrary to our assumptions, a user of our framework has good cause to believe that many interactions are weak and the network is instead sparse, there are multiple approaches for dealing with this issue. For example, setting strongly informed priors centred around zero for specific interactions that are thought to be negligible would be relatively easy to implement in the code we provide. More general alternative methods which have been developed explicitly for sparse interaction networks already exist, albeit where sparsity is applied across all interactions \parencite{Weiss-Lehman2022}. Forbidden links are a subset of potential interactions which cannot be observed, often due to physical constraints (e.g.\ biological mismatch) or spatio-temporal uncoupling. For example, a pair of short-lived annual plants might have non-overlapping growing seasons and thus never overlap in the field. We direct the reader towards the literature on forbidden interactions \parencite{Olesen2011, Jordano2016} for solving these cases. 

    \subsection{Case study}

       \paragraph{}
        We applied this framework to an annual wildflower community dataset from Western Australia \parencite{Bimler2023}, collected in 2016 (permit number SW017856 for "Flora take for a Scientific or Other Prescribed Purpose License" issued by the Western Australia Department of Parks and Wildlife). This dataset contains over 5000 observations of individual plant seed production from 22 different focal species, and the identity and density of all neighbouring individuals within a 3 to 5 cm radius of the focal individual. The environmental heterogeneity of this system is well studied \parencite{Dwyer2015} and we thus know that at the local scales across which this system was surveyed, only soil phosphorous, shade and the presence of woody debris impact on diversity and abundance patterns. To account for this known environmental heterogeneity, we randomly thinned plots to decouple abundance and environment effects (Supplementary Methods \ref{SI:comdata}). We used our framework to quantify interactions between these 22 focal species and 52 neighbour species, and derived scaled interaction strengths with a well-supported population dynamics model for annual plants with a seed bank \parencite{Levine2009, Bimler2018} which required experimentally-measured species demographic rates (Supplementary Methods \ref{SI:germination}). Viable seed production was used as the measure of performance and modelled with a negative binomial distribution and a log link. Further details on the procedure for deriving scaled interactions from the population dynamics model are available in the Supplementary Methods \ref{SI:popdyn}.

        \paragraph{}
        We fit all three models to the case study data with 4 chains and 7000 iterations, of which the first 5000 were discarded. All parameters for the NDDM-only model converged well; however, some of the $\gamma$ and $B$ parameters for the RIM-only and joint models received $\hat{R}$ values over $1.01$. High $\hat{R}$ values without other warnings are commonly associated with posteriors that are highly correlated and whose geometry is hence difficult to traverse \parencite{Team2022}. The impact of such ``problematic geometries'' however, is dependent on the data at hand, as evidenced by reliable convergence of all models on the simulated data. We explain why this may be the case for the $r$ and $e$ parameters in our case study and how this affects the $\hat{R}$ of $\gamma$ and $B$ in the Supplementary Methods \ref{SI:convergence}. By comparing the fits across models and the resulting posterior distributions of model parameters, we remained comfortable using the joint model estimates returned by multiple chains to further study the interactions underlying the case study data. Supplementary Methods \ref{SI:convergence} goes into further detail in this regard and, in particular, shows how the posterior intervals of the model parameters remain informative despite high $\hat{R}$ values. Model parameters were sampled 1000 times from the 80\% posterior confidence intervals to construct our parameter estimates. We then applied bootstrap sampling from each resulting interaction strength distribution to create 1000 samples of the community interaction network.

\section{Results}
    
    \paragraph{}
    The joint model framework returns a matrix $B$ whose elements quantify the effects of interacting species $j$ (columns) on the performance of focal species $i$ (rows). The interactions (values) which make up $B$ can be positive or negative, non-symmetrical (the effect of element $i$ on $j$ does not necessarily match the effect of element $j$ on $i$) and include intraspecific effects (the effect of element $i$ on itself). We illustrate the advantages of this approach in the case study results below. 


    \subsection{Case study results}

    \paragraph{}
    The model returned estimates for all 1144 interactions between 22 focal species and 52 interacting species, of which 56.7\% were identifiable and estimated by the NDDM. When accounting for interactions between focal species only, 82.0\% of interactions were identifiable. We conducted a posterior predictive check comparing simulated performance data from the joint model to observed values (Fig. \ref{fig:ppcheckmu2}). This is especially important for verifying that the appropriate distribution and link function are being used for the data at hand. The joint model also returns simulated performance data for the RIM only, which we also checked visually (Supplementary Results \ref{SI:results} Fig. \ref{fig:ppcheckmu1}). Model parameters were sampled 1000 times from the 80\% posterior confidence intervals returned by STAN to construct our parameter estimates. Interaction estimates between focal species were scaled according to the annual plant population model (Supplementary Methods \ref{SI:popdyn}) into interaction strengths affecting population growth, and we sampled from each resulting interaction strength distributions to create 1000 samples of the scaled community interaction matrix.

    \paragraph{}
    For our case study, the resulting scaled community interaction matrix was non-symmetrical and included both positive (competitive) and negative (facilitative) values, as shown in Figs. \ref{fig:netwks}.A \& B. Here we represent the community matrix as a network between all 22 focal species, taking the median value of each scaled interaction across all samples. Overall, the median strength for 55.8\% of focal $\times$ focal interactions were competitive, making competition the dominant interaction type. The median of 44.2\% of focal $\times$ focal interactions, however, were facilitative. As a result, the median of 47.6\% of interactions between pairs of focal species were of opposing signs such that $i$ competes with $j$ but $j$ facilitates $i$. Furthermore, the elements of the diagonal (the effect of a species on itself) were able to be estimated, which allows us to quantify how much a species regulates its own performance. Median intraspecific interaction strength was competitive for $13$ focal species (59.1\%), and facilitative for the remaining $9$ (40.9\%). For 11 of our 22 focal species, the scaled distributions of intraspecific effects did not overlap with $0$, suggesting that individuals of those species have a non-trivial effect on other individuals of the same species. When considering interspecific interactions with neighbouring focal species, the proportion which did not overlap with $0$ dropped to 25.5\%.

    \subsection{Examples of ecological applications}

    \paragraph{}
    We illustrate a few potential applications of our framework by exploring questions of common ecological relevance with our case study. Each question below highlights some of the advantages of our resulting interaction network: intraspecific interactions, non-symmetrical interactions, and the ability to estimate positive and negative interactions. 

    \subsubsection*{Q1. Do abundant natives under-regulate their population density compared to rarer native species?}
    One hypothesis as to why certain plant species are more abundant than others is that they tend to compete with themselves less strongly than rare species \parencite{Yenni2012, Yenni2017}. Hypothetically, this release from intraspecific competition pressure allows them to reach much higher densities than species which strongly compete with themselves. In our case study, we explore this hypothesis by plotting the effect of a species on itself against its density as in Fig. \ref{fig:species}.A. Intraspecific interactions are at their weakest when close to $0$. The two most abundant native species \textit{Velleia rosea} (VERO) and \textit{Podolepsis canescens} (POCA) highlighted in purple fall very close to the median intraspecific interaction strength. This suggests that \textit{V. rosea} and \textit{P. canescens} do not reach high densities through an under-regulation of their population density and are thus likely to be reaching these densities through other means such as access to a larger niche space. 

    \subsubsection*{Q2. Which species are keystones in this system?}
    Keystone species have disproportionately strong effects on other species in their system and the dynamics of the whole ecosystem, based on their abundances \parencite{Power1996, Piraino2002, Libralato2006}. As such their exclusion from a community is expected to create significant changes in species density and composition \parencite{Paine1969}. Though determining which species truly serve keystone roles has historically involved extensive ecological experimentation (e.g.\ \cite{Paine1992}) and the inclusion of multiple trophic levels, we can identify potential candidates by comparing a species' impact on the population growth of other species to its own density \parencite{Libralato2006}. It is important to note that because our framework allows for asymmetrical interactions, we are able to differentiate a species' impact on other species from its response or sensitivity to neighbours \parencite{Broekman2020}. Fig. \ref{fig:species}.B highlights a native species in green, \textit{Gilberta tenuifolia} (GITE), which may be a potential keystone species due to having strongly competitive effects on the rest of the community overall, despite low density. 

    \subsubsection*{Q3. Do all exotic species compete with native species?}
    Though invasive species are expected to compete with natives \parencite{Naeem2000, Corbin2004, Riley2008, Zheng2015}, several studies have found evidence of exotic species facilitating natives, with cascading effects on other species and net positive effects on ecosystem processes \parencite{Rodriguez2006, Ramus2017, Wainwright2019}. By allowing for positive and negative interaction strengths between species in a system, we can determine which exotics are harmful or beneficial to the native species in a community. Fig. \ref{fig:species}.C plots the sum of a species' competitive effects on neighbours against the sum of its facilitative effects on neighbours. Exotic species are identified in red. Out of these three exotic species, two have overall competitive effects on the community: \textit{Arctotheca calendula} (ARCA) and \textit{Pentameris aroides} (PEAI). Both species have weak or close-to-median effects on their neighbours compared to other focal species. \textit{Hypochaeris glabra} (HYPO) on the other hand has strong effects on other species, both competitive and facilitative, and has an overall facilitative contribution to the community. These effects are partly driven by its incredibly high germination rate (over twice as high as any other focal species). These results suggest that here at least, the effects of exotic species on native species are complex and species-dependent.


\section{Discussion}


    \paragraph{} 
    Our novel framework quantifies the effects of interacting species and reciprocal performance, allowing the estimation of diverse, horizontal interaction matrices. The resulting matrices are non-symmetrical and can contain both positive and negative interactions, as well as the effect of a species on itself. This framework is flexible to metrics of performance, type of group (e.g.\ species, population, etc.) and diversity. We also propose a way to approximate unidentifiable interactions given information about those which are identifiable, which is a significant feature given that networks for horizontal systems are expected to be non-sparse. The interaction matrices generated through this framework can be transformed into interaction networks through the use of models describing the system's interaction dynamics. Together, these features differentiate our framework from other methods currently available for estimating interactions in diverse horizontal systems and make it particularly useful in an ecological context (as illustrated in our case study) as well as flexible for use with data from the wide range of complex systems dominated by horizontal interactions.

    \paragraph{}
    In particular, our approach includes a method to infer all pairwise interactions despite `incomplete' data. There are, generally speaking, two alternative strategies to deal with this issue, both of which attempt to reduce the number of interactions to be estimated. The first assumes that many species have similar effects on one another and can be grouped a priori according to biological factors (e.g. traits, life form) \parencite{Uriarte2004, Martyn2020}. The second assumes that a majority of interactions are weak and hence can be removed from the model through variable selection procedures \parencite{Mutshinda2009, Weiss-Lehman2022}, resulting in a sparse interaction network. \textcite{Weiss-Lehman2022} defined interactions as a combination of an average community-level measure and species-specific deviations from this average, and they used regularisation approaches to allow only a subset of these deviations to take non-zero values. In our case study, many interaction estimates overlapped with zero, though as discussed in the Methods \ref{meth:sparse} this does not necessarily imply sparsity.

    \paragraph{}
    \textcite{Ovaskainen2017} present a similar approach to ours but for time-series data. Their method assumes that interspecific interactions can be described by a small number of community-level drivers (effectively linear combinations of species abundances) which best predict future growth rates. Though this method requires all intraspecific interactions to be directly inferrable, the number of community-level drivers is much smaller than the number of species, allowing interspecific interactions between many species to be quantified despite relatively short time-series data. They compared their framework to both sparse and full interaction models, the latter performing poorly due to overfitting. Indeed, whether our framework provides a better fit to data remains to be tested, and will depend strongly on the particulars of both the data and system at hand. Nonetheless, rather than treat sparse and full interaction models as an either-or question, our joint method provides an `intermediate' way to make use of data that has historically been insufficient or too incomplete to infer all pairwise interactions. We thus expect it will open up a wide range of questions that were previously difficult or impossible to answer. Though we illustrate our study with one particular ecological dataset, the method presented here could be adapted for use on a wider variety of horizontal systems such as those found in microbial, neural, and social networks. 

    \paragraph{}
    Species interaction networks have a wide range of practical applications, such as evaluating ecosystem response to human-altered landscapes, guiding future management decisions \parencite{Ross2011} or exploring how communities may respond to global warming \parencite{Gorman2019}. Conservation and ecosystem management efforts aimed at regulating species abundances can, for example, use the information provided by an interaction network to prioritise which species to conserve or eradicate based on their role in the community \parencite{Cirtwill2018a}. Identifying keystone, foundation and other important types of species roles is also helpful for understanding biological diversity, ecosystem integrity and functioning, especially in response to disturbances and other stresses \parencite{Nyakatya2008, Orwin2016, Losapio2017, Narwani2019} though often requires the inclusion of other trophic levels. The examples we describe here are not exhaustive, but serve to illustrate how horizontal interaction networks, especially when linked to population models, can help us understand both community dynamics overall and the effects \& response of specific species towards the community. 

    \paragraph{}
    Quantifying the community interaction matrix can also allow us to explore how the mechanisms maintaining diversity and stability operate in these systems and across a broad number of species. Self-regulation, for example, is an extremely important driver of community stability \parencite{Barabas2017} and arises from how individuals of the same species interact with one another. Measures of intra and interspecific interactions can also allow us to estimate niche overlap between species (for an example, see \cite{Chu2015}); weak interactions between species suggest that they are not sharing or competing for many resources, and thus may have large niche differences in the community. Moreover, our inclusion of facilitative interactions, which have traditionally been disregarded in plant population models and mathematical frameworks of plant coexistence, provides a means to investigate the prevalence and strength of facilitation across multiple species and how it may act in relation to competition and species diversity. Recent work suggests facilitation may be more widespread than traditionally thought \parencite{Gross2015, Picoche2020} and is likely to benefit species diversity and stability in some systems \parencite{Coyte2015, Brooker2008}.

    \paragraph{}
    Ultimately, quantifying species interaction networks allows us to apply tools from network theory to help us understand how these interactions drive community-level patterns of density and diversity. Several metrics already exist for describing horizontal network structure such as weighted connectance \parencite{Ulanowicz1991} or relative intransitivity \parencite{Laird2006}, though these are fewer than for trophic or unweighted networks (e.g.\ \cite{Bersier2002, Delmas2019}). Adapting measures of nestedness or modularity for example to non-sparse networks (as horizontal communities typically are) would allow us to further characterise how interactions and species are organised. These metrics relate to various aspects of stability and could greatly inform us on how diversity is maintained. Likewise, networks also provide several ways of measuring and describing species roles in their communities \parencite{Cirtwill2018a} for example through the use of structural motifs, unique patterns of interacting species which together make up the whole network. Motifs have been found to have important biological meaning in food webs \parencite{Bascompte2005a} but remain to be identified for horizontal networks. 

\section*{Acknowledgements}

We thank M. Raymundo \& I. Towers for the seed rate data used in this paper, C. Bowler, T. Britton, and J. Ikin for their work in sample processing for this data set, as well as Rob Freckleton, Jacopo Grilli, Stefano Allesina and two anonymous reviewers for insightful comments on the manuscript. We also wish to acknowledge the Traditional Custodians of the Country from which the case study data was collected, the Yamatji People, as well as those of the land on which this research was conceived of, carried out, and written, the Jagera and Turrbal Peoples. This work was made possible by funding awarded to M.M.M. (DP170100837) by the Australian Research Council. D.B.S. is grateful for the support of the Marsden Fund Council, from New Zealand Government funding (grant 16-UOC-008).


\section*{Conflict of Interest}

The authors declare no competing interests. 

\section*{Author contributions}

Malyon D. Bimler designed the methodology, carried out analyses and led the drafting of the manuscript. Margaret M. Mayfield helped design the field study, collected data, contributed to the interpretation of analyses and critically revised the manuscript. Trace E. Martyn led the field study and data collection. Daniel B. Stouffer helped design the methodology, interpret analyses and critically revised the manuscript. 

\newpage

\printbibliography   

\end{refsection}

\newpage 

\section{Figures}


    % FIGURE 1

    \begin{figure}[H]
        \includegraphics[width=.6\textwidth]{../../3.analyses/figures_mss/postpredch_mu2.png}
        \caption{Posterior predictive check showing the density distribution of observed seed production values (red line) to simulated seed production values (light grey) as estimated by the joint model, on a log scale. Simulated values were generated using the 80\% posterior confidence intervals for each parameter, the black line shows simulated values using the median of each parameter. }
        \label{fig:ppcheckmu2}
    \end{figure}

    % FIGURE 2

    \begin{figure}[H]
        \begin{centering}
        \includegraphics[width=0.5\textwidth]{../../3.analyses/figures_mss/networks_C_F.png}
        \caption{Competitive (A) and facilitative (B) scaled interaction networks estimated from our model framework. Competitive and facilitative interactions (A and B) are shown separately for ease of viewing but were analysed together. Only focal species are included in these networks, arrows point to species $i$ and line thickness denotes interaction strength. Interaction strengths are given as the median over 1000 samples. Purple coloured nodes correspond to highly abundant native species, red nodes to exotic species, and the green node indicates a potential keystone species.}
        \label{fig:netwks}
       \end{centering}
    \end{figure}  

    % FIGURE 3

    \begin{figure}[H]
        \begin{centering}
        \includegraphics[width=0.4\textwidth]{../../3.analyses/figures_mss/species_effects.png}
        \caption{(Caption next page.)}
        \label{fig:species}
        \end{centering}
    \end{figure} 

    \addtocounter{figure}{-1}
    \begin{figure} [t!]
        \caption{For all graphs, diamonds are species medians across all network samples, black lines cover the 50\% quantile and grey dots indicate the full range of values as calculated from 1000 sampled networks. Note that only focal x focal interactions are included. The shaded part of the graphs show facilitative (negative) interactions. Dashed lines represent the median value for all focals. Coloured triangles indicate the species referred to in the main text for each of the ecological questions associated with (A), (B) and (C). \\
        In (A), the x-axis shows the strength of scaled intraspecific interactions, that is how strongly a focal species interacts with itself, plotted against a focal species' total log abundance (y-axis). Values above $0$ indicate competition, and values less than $0$ indicate facilitation.  The two most abundant natives, \textit{Velleia rosea} (VERO) and \textit{Podolepsis canescens} (POCA) (purple triangles), do not compete with themselves any more or less strongly than the median for all species in the system (dashed line). \\
        (B) shows the sum of interaction effects of focal species on neighbours (x-axis) against the focal species' total log abundance (y-axis). On the x-axis, values greater than $0$ indicate that a focal species has an overall competitive effect on neighbours, and values less than $0$ indicate that it has an overall facilitative effect. Green triangles identifies a species with low overall abundance but strong competitive effects on neighbours: \textit{Gilberta tenuifolia} (GITE). \\
        (C) decomposes a focal species' net interaction strength into its competitive effects (x-axis) and facilitative effects (y-axis). Red triangles show the exotic species \textit{Hypochaeris glabra} (HYPO), \textit{Arctotheca calendula} (ARCA) and \textit{Pentameris aroides} (PEAI). The light grey diagonal shows where x = y, species above that line have an overall facilitative effect on other species whereas those below that line have an overall competitive effect. Points capturing the effects of \textit{H. glabra} on neighbours are spread much further apart than for any other species, this is largely driven by it's high germination rate which magnifies interaction effects when those are scaled according to the population dynamics model (see Supplementary Methods \ref{SI:popdyn}).} 
    \end{figure}

\clearpage
\newpage


% END OF MAIN TEXT 
\begin{refsection}

\section{Supplementary Methods}

\setcounter{figure}{0}

\label{SI:Methods}

    \subsection{Model code and parameterisation}
    \label{SI:modelcode}
    

    We make the several observations in addition to comments in the code. Firstly, identifiable interactions (beta\_ij in the code) are defined as a vector, which must then be matched to their correct position in the interaction matrix. This is the role of the istart, iend, icol, and irow vectors defined in the data block. Our github repository also contains the data\_prep.R file, which will show how to calculate these vectors from the input data. 
    Secondly, we impose the following constraints to improve convergence, avoid over-parameterisation and maintain identifiability of our parameters $r_{d_k}$ and $e_j$ \parencite{Huber2004, Kidzinski2020, Niku2021}. We define the effect parameters as a unit vector, which means we only require K-1 degrees of freedom (where K is the total number of neighbour elements) to estimate all effect values. This loss of a degree of freedom arises from the fact that the matrix of $r_{d_k}$ $e_j$ parameters is of rank 1. The first response parameter is also forced to positive. This improves convergence by providing an anchor for all other parameter values to 'rotate' around. Though these latter two have implications for our estimates of the latent variables $r_{d_k}$ and $e_j$, estimates for identifiable and unidentifiable interactions should not be affected.

    \subsection{Assessing parameter identifiability}
    \label{SI:identify}

    \textbf{We assess which interactions parameters are identifiable in the data\_prep.R file (lines 30-44) and store this information in the $Q$ matrix. To do so, we create a model matrix $X_i$ for each focal species $i$ which is made up of a column vector of 1's (representing the intercept) and the subset of neighbour abundances recorded for that focal species (one predictor aka neighbour species per column). We transform this matrix into its row-reduced echelon form $R_i$ using the rref() function from the pracma package \parencite{Borchers2022}. We then multiply $R_i$ by its transpose to get the symmetric matrix $Z_i$, which we can use to determine which predictors (rows) are linearly independent of all other predictors. Note that given $Z-i$ is symmetrical, this can also be done using the columns of $Z_i$. We ignore the first row (column) of $Z_i$ because it corresponds to the intercept which we always want to include. For every row (column) $k > 1$, it is linearly independent if that row (column) is a vector of $0$'s except for the k'th element. If this is true, the interaction parameter corresponding to that row is thus identifiable and we assign it a value of $1$ in the $i$'th row and $k-1$'th column of $Q$. Alternatively, the corresponding interaction parameter is given a value of $0$ in $Q$. Evaluating the $Z_i$ matrix of each focal species therefore allows us to construct the $Q$ matrix of inferrable parameters.}

    In a similar way, we also evaluate whether predictors (neighbours) are linearly independent across the entire dataset in order to be able to correctly distinguish between their respective effect parameters $e_j$ for the RIM. The same procedure as above is performed across the entire model matrix (without subsetting for each focal species) and neighbours are evaluated for linear independence. This is is done early on in the master.R file (lines 26-36) as a check before transforming the data into the format required by STAN. A lack of linear independence across the entire dataset is very unlikely, and would only arise if multiple neighbours' densities were perfectly correlated across all neighbourhoods.

% We construct a matrix $Q$ of size $s \times t$, with $Q_{i, j} = 1$ if the corresponding $\beta_{i, j}$ parameter is identifiable, and $Q_{i, j} = 0$ if not.

    \subsection{Validation on multiple chains}
    \label{SI:multichains}


    \textbf{To further validate good chain behaviour and convergence of our model, we conducted an additional model run using three MCMC chains, each run for a total of 4000 iterations with the first 2000 discarded. We confirmed good chain mixing through the visual inspection of traceplots. All chains converged, as evidenced by the distribution of the split $\hat{R}$ statistic  which did not exceed $1.004$ over all parameters. The posterior distributions returned from multiple chains matched closely to those returned from running the model on a single chain, we thus present results from the single chain procedure in the main text. 
    }

    \subsection{Case study Methods}
    \label{SI:casestudy}

        \subsubsection{Community data}

        \paragraph{}
        We applied this framework to annual wildflower community dataset from Western Australia. This system is a diverse and well-studied community of annual plants which germinate, grow, set seed and die within approximately 4 months every year. Individual fecundity data were collected in 2016, when 100 50 x 50 cm plots established in the understory of West Perenjori Reserve (29$^o$28'01.3"S 116$^o$12'21.6"E) were monitored over the length of the full field season. The resulting dataset includes between 29 to over 1000 counts of individual plant seed production from 22 different focal species (with a median of 108 observations per species), in addition to the identity and densities of all neighbouring individuals within the interaction neighbourhood of each focal plant. Interaction neighbourhoods varied in radius from 3 to 5 cm depending on the size of the focal species \parencite{Martyn2020}. Total neighbourhood diversity was 71 wildflower species, 19 of which were recorded fewer than 10 times across the whole dataset. The species-specific effects of this latter group of species on focals were deemed negligible due to their extremely low density, they were thus grouped into an 'other' category and their effects on focals averaged. This resulted in 53 potential neighbour identities. Half of all plots were thinned (a quarter to 60\% diversity and a quarter to 30\%) to mitigate possible confounding effects between plot location and plant density, and thinning did not target any particular species. 

        \paragraph{}
    	In study systems which allow it, a proportion of interaction neighbourhoods can instead be thinned prior to the experiment to randomly remove neighbouring individuals and provide observations for low-density estimates of interactions. Though this steps is not strictly necessary, thinning certain neighbourhoods can also reduce potential confounding effects between the environment and interactions and thus provide more accurate estimates of interaction effects. Environmental data known to affect performance can also be recorded and included in the model (as a random effect for example) to minimise those confounding effects.

        \paragraph{} 
        We required species demographic rates (seed survival and germination) in order to scale model interaction estimates into interaction strengths. Species demographic rates for 16 of our focal species were estimated from a database of field experiments carried out between 2016 and 2019 where seedbags were placed in the field to estimate germination rates, and ungerminated seeds were evaluated in the lab for survivability.  The remaining species were assigned mean demographic rates from these experiments. Further details on the methods used for collecting those seed rates are available in section \ref{SI:germination}.

        \subsubsection{Model fitting}

        \paragraph{}
        We fit the model using R version 3.6.3, STAN and the rstan package \parencite{R2020, Carpenter2017, Rstan2020}. Estimates of seed production were fit with a negative binomial distribution. The model was run with 1 chain of 10000 iterations, discarding the first 5000. Models were checked for convergence using the geweke.diag() function from the coda package \parencite{Plummer2006} and traceplots were visually inspected to verify good chain behaviour. Model parameters were sampled 1000 times from the 80\% posterior confidence intervals to construct our parameter estimates. We then applied bootstrap sampling from each resulting interaction strength distribution to create 1000 samples of the community interaction network.

        \subsubsection{A model for annual plant population dynamics}
        \label{SI:popdyn}

        \paragraph{}
        The above model framework returns species-specific estimates of intrinsic fitness ($\gamma_{d_k}$), as well as as a species x neighbour matrix $B$ of identifiable ($\beta_{d_k, j}$) and unidentifiable ($r_{d_k} e_j$) interaction estimates which quantify the effects of one neighbour $j$ on the intrinsic fitness of a focal species $i$. Though useful as they are, these estimates can lead to a wider range of potential applications when integrated into models of population dynamics. For example, we might be more interested in the effects of neighbours on the density or growth rate of a focal species rather than on it's proxy for lifetime reproductive success. Importantly, it is necessary to specifiy a model describing population dynamics in order to draw conclusions about the effects of interactions and network structure on the maintenance of community diversity and stability. 

        \paragraph{} 
        We defined the following model for annual plants with a seed bank \parencite{Levine2009, Mayfield2017, Bimler2018} which describes the rate of change in a focal species' \textit{i} abundance of seeds in a seed bank from one year to the next: 
            \begin{equation}
                \frac{N_{i, t+1}}{N_{i, t}} = \left( 1 - g_{i} \right) s_{i} + g_{i}F_{i, t}
                \label{ifm}
            \end{equation}
        where \(F_{i,t}\) measures the number of viable seeds produced per germinated individual whilst \(g_{i}\) and \(s_{i}\) are the seed germination and seed survival rate, respectively. In a simplified case where the focal species \textit{i} interacts with only one other species \textit{j}, our use of a log link function implies that \(F_{i,t}\) in this model of population dynamics is given by:
            \begin{equation}
                F_{i,t} = \lambda_{i} e^{- \alpha_{ii} g_{i} N_{i, t} -  \alpha_{ij} g_{j} N_{j, t} }
                \label{fecundity}   
            \end{equation}
        where \(\lambda_{i}\) corresponds to seed number in the absence of interaction effects, and \(\alpha_{ii}\) and \(\alpha_{ij}\) are the interaction strengths between species \(i\) and its intraspecific and interspecific neighbours respectively. Here it is \(\alpha_{ij}\) and \(\alpha_{ii}\) which are equivalent to \(\beta_{d_k, j}\) in Eq.~\ref{nddm} of the main text. 
        We determine the scaled interaction strengths ${\alpha}''$'s by including \(\lambda_{i}\), \(g_{i}\) and \(s_{i}\) in such a way that these variables are cancelled out when the ${\alpha}''$'s are substituted for the $\alpha$'s in our annual plant population model \parencite{Godoy2014, Bimler2018}. 
        \begin{equation}
            {\alpha}''_{ij} = \frac{g_{j} \alpha_{ij}}{ln(\eta_{i})}
        \end{equation}
        with $\eta_{i} = \frac{\lambda_{i} g_{i}}{\theta_{i}}$ and $\theta_{i} = 1 - (1 - g_{i})(s_{i})$. % \(ln(\eta_{i})\) is thus equivalent to \(\beta_{i0}\) in Eqs. \ref{nddm} and \ref{scaling}.
        Note that our model evaluates the rate of change of seeds in the seed bank, and this is reflected in the scaling terms used to compare interaction strengths between focal species. Substituting ${\alpha}''$'s for $\alpha$'s in Eq.~\ref{ifm} gives us: 
    
        \begin{equation}
            \frac{N_{i, t+1}}{N_{i, t}} = (1 - \theta_{i}) + \theta_{i} \eta_{i} e^{-ln(\eta_{i})({\alpha}''_{ii} N_{i, t} + {\alpha}''_{ij} N_{j, t})}
        \end{equation}
        where we can see that the ${\alpha}''$'s are directly proportional to the density of neighbours. Relating this population model to the joint model framework, we recover the following: 

        \begin{equation}
        {\beta}''_{d_k, j} = {\alpha}''_{ij}
        \end{equation}

        \begin{equation}
        \beta_{d_k, j} = \alpha_{ij}
        \end{equation}

        \begin{equation}
        \gamma_{d_k}  = ln(\eta_{i}) = ln(\frac{\lambda_{i} g_{i}}{\theta_{i}})
        \end{equation}
        % Daniel thinks this last equation is wrong but I don't think so?
        As we show here, the exact form of the rescaled interactions as well as intrinsic fitness can therefore vary depending on the specific population dynamic model applied and may include other demographic rates which reflect species-level differences in growth and mortality. Because intrinsic fitness is estimated by the model framework and not directly observed, we used the mean of the $\gamma_{d_k}$ posterior distribution returned by our model in our scaling of the interaction coefficients.


        \subsubsection{Seed germination and survival data}
        \label{SI:germination}

        \paragraph{}
        Seed demographic rates were collected from a set of field experiments conducted by T. Martyn, M. Raymundo and I. Towers at Perenjori reserve between 2015 and 2019. Experiments differed both in the methods and in which focal species were included in ways which are detailed further below, such that each focal species had a different number of replicates across all experiments. Given how much seed rates estimates have been found to vary within species and according to a range of both individual and environmental factors, we chose to average results from these multiple experiments for each focal species in order to provide a point estimate which captures a wide range of conditions under which seeds may grow. For those species which did not have any field estimates of seed rates (\textit{Austrostipa elegantissima}, \textit{Erodium sp.}, \textit{Petrorhagia dubia}) or seed survival rate (\textit{Gilberta tenuifolia}), no replication (\textit{Waitzia acuminata}) or an unrealistically low estimate of germination rate (\textit{Goodenia pusilliflora}), we substituted the community mean instead. 

        \paragraph{}
        For each experiment, mature seeds were collected at the end of the growing season (September - October) from multiple populations of each focal species located throughout the reserve. Immature or damaged seeds were not included, and collected seed was homogenised for each focal species to elimiate bias associated with local adaptation within populations. Germination rate was estimated by planting seeds in the field along gradients of soil phosphorus, woody canopy cover and herbaceous vegetation density and either directly counting the number of seeds which had germinated or comparing recruitment rates to unplanted plots after a sufficient amount of time had elapsed. Seeds were planted during late-September to mid-October, mimicking natural seed dispersal timing for wildflowers in the area. Seed surival rates were estimated using either the remaining seeds or a separate batch of seeds and assessing viability of the seeds using tetrazolium staining. 


        \paragraph{T. Martyn experiment:}
        For 19 focal species (the full species list excluding \textit{A. elegantissima}, \textit{P. dubia} and \textit{G. tenuifolia}), germination bags containing 20 seeds each were planted in the field in 2016 across multiple areas of Perenjori Reserve. Out of the 30 bags, 19 were collected in 2017 and the remaining were collected in 2018. Due to a severe drougth in 2017, half of the bags collected that year were watered during the field season and prior to collection. Germination bags were then brought back to the Mayfield Lab facilities at the University of Queensland, Brisbane, and seeds extracted. Seeds were examined for signs of germination in the field (broken or empty seed coat) and those remaining were placed in germination trays and a germination chamber to mimic light and temperature conditions conducive to germination. Trays were watered with Gibberellic acid once to twice a week and seedlings were recorded and removed until no more seedlings emerged. Remaining, ungerminated seeds were then assessed as dead (moldy) or potentially viable. The remaining potentially viable seeds were assessed for viability using tetrazolium staining, contributing to our estimates of seed survival rates. For this procedure, embryos in each seed were exposed by either removing the seed coat or by creating a thin cut along the seed coat. The exposed embryos were then placed on a six-well germination plate and 2 ml of 0.25\% Tetrazolium solution was added to each well to stain the embryos, before covering them and storing them at 25$^{\circ}$C overnight. To check for staining, embryos were dissected under a dissecting microscope. Viable seeds showed a dark pink embryo while non-viable seeds did not stain or were stained in a splotchy way.


        \paragraph{M. Raymundo experiment:}
        This experiment was carried out on the focal species \textit{H. glutinosum}, \textit{T. cyanopetala}, \textit{T. ornata} and \textit{V. rosea} from 2015 to 2017. However, a severe drought in 2017 made the second round of data collection impossible and thus we only include results for 2016 here. Ten plots were established measuring 0.5 m x 0.5 m at each of three sites in Perenjori Reserve for a total of 30 plots. Each plot was divided into 25 0.1 m x 0.1 m subplots and focal species were randomly assigned a subplot in each plot. Thirty seeds of each focal species were planted in the designated subplot in late September 2015 and a plastic ring 10 cm in diameter and 1 cm high was placed in each subplot where seeds were added to limit seed movement among subplots. Another five subplots were assigned plastic rings to serve as controls for the effect of the rings on non-experimental communities. The remaining 15 subplots served as controls where no seeds or rings were added allowing for recruitment from either natural dispersal or from the seed bank. Blocks were placed in such a way as to span shaded and open areas, bare ground and dense herbaceous vegetation, and areas with native dominated and exotic dominated assemblages. Before implementing the experiment in 2015, plots were surveyed to record the number and identity of all adult plants in each subplot. Due to the randomization of seed addition into subplots, some subplots had focal species already in them. As all focal species were common to this reserve, it is also likely that seeds for all species were in the seed banks in at least some subplots. There was no way to determine this in advance, though when adult individuals of a focal species were present in a subplot prior to the implementation of our experiment, we expected that some seedlings in the following year would be from the seedbank as well as our planted seeds and looked for evidence of this (more than 30 individuals) in data from 2016. We therefore compared average densities of successful focal recruits and those which emerged in situ between sown, control, and ringed subplots to assess seed limitation and germination rate. To measure seed survival rates, thirty seeds of each focal species were also assessed for viability using tetrazolium staining using the same procedure as for the T. Martyn experiment. 

        \paragraph{I. Towers experiment: }
        This experiment was carried out on the focal species \textit{A. calendula}, \textit{G. berardiana}, \textit{H. glutinosum}, \textit{H. glabra},\textit{P. aroides}, \textit{P. debilis}, \textit{P. canescens}, \textit{T. cyanopetala}, \textit{T. ornata}, \textit{V. rosea} and \textit{W. acuminata} in 2018 and 2019. Pairs of free-draining germination trays were deployed across a gradient of canopy cover in mid-October of both years, filled with soil, which had either been collected from the field and heat-sterilised to render pre-existing seeds nonviable (2019), or simply collected from the roadside (2018). Each germination tray consisted of 24 cells, with two cells randomly assigned to each focal species. In each cell, 15 seeds of the designated focal species were broadly distributed and lightly misted with water to facilitate seed-soil contact and minimise removal by wind. Trays placed in 2018 used seeds collected at the end of the 2017 growing season and dry after-ripened at 60$^{\circ}$C for a month before being stored in cool, dry conditions at the University of Queensland. Seeds planted in 2019 were collected at the end of the 2018 growing season and were placed directly from the field into the germination trays. To re-establish microbial communities for those trays where the soil had been heat-treated, seeds were lightly covered with a small amount of untreated soil collected from the site in which they were buried. Untreated soil was collected from directly underneath coarse woody debris in patches where it was present as prior research in this system has shown that the effect of coarse woody debris on plant performance is partially attributable to debris-specific soil microbial communities (A. Pastore unpublished data). Some of the trays received the additions of leaf litter, but the results of this treatment were not included for the seed rates used in this study. Seed germination rate was measured by counting the number of seedlings which emerged in the field, but seed survival rate was not calculated in this experiment.


        %\subsection{References}
        \printbibliography   
        % Add Supps bibliography here!
\end{refsection}


\section{Supplementary Results}
\label{SI:results}


    \begin{figure}[H]
       % \hspace*{-3.5cm}
        \includegraphics[width=.6\textwidth]{../../3.analyses/figures_mss/postpredch_mu.png}
        \caption{Posterior predictive check showing the density distribution of observed seed production values (red line) to simulated seed production values (light grey) as estimated by the RIM only, on a log scale. Simulated values were generated using the 80\% posterior confidence intervals for each parameter, the black line shows simulated values using the median of each parameter. }
        \label{fig:ppcheckmu1}
    \end{figure}

    \begin{figure}[H]
       % \hspace*{-3.5cm}
        \includegraphics[width=\textwidth]{../../3.analyses/figures_mss/interaction_estimates.png}
        \caption{Distribution of interaction estimates from our case study. Parameter estimates are sampled from the 80\% posterior confidence intervals returned by STAN. Upper left panel shows the distribution of observed interactions as estimated by the NDDM ($\beta_{d_k, j}$), which are then plotted against the corresponding RIM estimates ($r_{d_k} e_j$, x-axis) in the upper right panel. Bottom rows show the distribution of \textit{unidentifiable} (left) and identifiable interactions estimates returned by the RIM. Interaction estimates are unscaled.}
        \label{fig:adist}
    \end{figure}

   % NETWORK FIGURES

    \begin{figure}[H]
        \includegraphics[width=\textwidth]{../../3.analyses/figures/joint_network.png}
        \caption{The full network of unscaled interactions between 22 focal species as estimated by the joint modelling framework presented here. Arrows point to species $i$ and line thickness denotes median interaction strength over 1000 samples drawn from the 80\% posterior confidence interval of each parameter. Facilitative interactions are shown in blue and competitive interactions are shown in yellow.}
        \label{fig:jointnetwork}
    \end{figure}

    \begin{figure}[H]
        \includegraphics[width=\textwidth]{../../3.analyses/figures/NDDM_network.png}
        \caption{Network of identifiable interactions (unscaled) between 22 focal species as estimated by the NDDM, unidentifiable interactions are not shown here. Arrows point to species $i$ and line thickness denotes median interaction strength over 1000 samples drawn from the 80\% posterior confidence interval of each parameter. Facilitative interactions are shown in blue and competitive interactions are shown in yellow.}
        \label{fig:jointnetwork}
    \end{figure}

    \begin{figure}[H]
        \includegraphics[width=\textwidth]{../../3.analyses/figures/RIM_noninferrables_network.png}
        \caption{Network of unidentifiable interactions (unscaled) between 22 focal species as estimated by the RIM. Arrows point to species $i$ and line thickness denotes median interaction strength over 1000 samples drawn from the 80\% posterior confidence interval of each parameter. Facilitative interactions are shown in blue and competitive interactions are shown in yellow.}
        \label{fig:jointnetwork}
    \end{figure}



\end{document}
