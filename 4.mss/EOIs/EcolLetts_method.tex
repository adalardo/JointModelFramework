% Method are commissioned by unsolicited proposals, which will be evaluated by the Perspective or Synthesis or Method Editors, in consultation with the Editorial Board and Editor-in-Chief, prior to a full submission. Proposals should be no more than 300 words long, describe the nature and novelty of the work, the contribution of the proposed article to the discipline, and the qualifications of the author(s) who will write the manuscript. Proposals should be sent to the Editorial Office (ecolets@cefe.cnrs.fr).

% Method reports on new instrumentation, experimental procedures and statistical models that are expected to catalyse significant breakthroughs across ecology's sub‐disciplines. Papers should contain sufficient information to transfer methods to new settings, systems, and research questions. For instance, methods using unique reagents must explain how the new formulation may be acquired; quantitative methods must be accompanied by high quality code. All Method papers should focus on the method itself, but demonstrate use in a case study. The review criteria is based on the novelty and importance of the method.

% _____________________________________________________________________________

\documentclass{letter}
\signature{Dr. Malyon Bimler, Prof. Margaret Mayfield, Dr. Trace Martyn \& Dr. Daniel Stouffer}
\address{School of Biological Sciences \\ University of Queensland \\ St Lucia \\ Australia}

\begin{document}

\begin{letter}{Editorial Office \\
Ecology Letters}

\opening{}

Please find below our proposal for our manuscript titled: \textit{Estimating interaction matrices for diverse, horizontal systems}, which we would like to submit for publication at Ecology Letters as a Method. 

\textbf{Proposal:}
% describe the nature and novelty of the work, 
% the contribution of the proposed article to the discipline, 
% and the qualifications of the author(s) who will write the manuscript.
Network theory allows us to understand complex systems by evaluating how their constituent elements interact with one another. It has been widely applied to many disciplines in biology, including ecology, where there is now a rich body of work describing the characteristics of food webs and other trophic interactions. Fewer network approaches, however, focus on non-trophic or horizontal interactions occurring between elements on the same organisational level, for example between plants. Indeed, quantifying these interaction matrices from empirical data is challenging, as the number of potential interactions increases non-linearly as more elements are included, and not all interactions may be empirically observable when some elements are rare. 

We present a novel modelling framework which estimates observed and unobserved interactions in diverse horizontal systems, using measures of individual performance in the absence and presence of their potential interaction partners. This method is illustrated with a case study on a diverse wildflower community, and is flexible to positive and negative interactions, as well as to a wide range of data from other horizontal complex systems. In turn, this framework facilitates the application of network theory tools to the management of horizontal systems, as well as to deepening our understanding of diversity, stability and other system-level properties which emerge from interactions. 

The manuscript is authored by Dr Malyon Bimler, a postdoc at the University of Queensland who's doctoral thesis focused on merging theoretical and empirical approaches to understanding species diversity in natural systems, Professor Margaret Mayfield, an international expert in Plant Community Ecology, Dr Trace Martyn, a postdoc in the School of Natural Resources and the Environment at the University of Arizona, and Associate Professor Daniel Stouffer, currently leading the Complex Systems Ecology group at the University of Canterbury.

% The lead author, Dr Malyon Bimler, is currently a research assistant at the University of Queensland prior to starting next her post-doctoral appointment at UQ. Her doctoral thesis focused on merging theoretical and empirical approaches to understanding species diversity in natural systems, during which she developed the method described in our proposal. Under the guidance of Dr Daniel Stouffer, Associate Professor leading the Complex Systems Ecology group at the University of Canterbury, and Professor Margaret Mayfield, an international expert in Plant Community Ecology, 


\newpage

We thank you for your consideration of our proposal and look forward to receiving your feedback.

\closing{Sincerly,}

% \ps{P.S. Here goes your ps.}

% \encl{Enclosures.}

\end{letter}
\end{document}

